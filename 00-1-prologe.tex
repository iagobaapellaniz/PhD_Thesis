\section*{Prologue}

\lettrine[lines=2, findent=3pt,nindent=0pt]{T}{his} work is part of the doctoral project on quantum information which I started on the summer of 2013.
It collects part of the research I have done on these previous years.
I will try to be clear throughout the thesis such that make it useful for a wider audience as it is possible.
This way I hope it will be readable by any person with a bachelor in science, particularly in physics or at least with a partial knowledge in quantum mechanics.
With that in mind, the first and the second chapters will be used to introduce the reader into the context in which this thesis was written as well as the basic notions of quantum metrology, the enveloping field of the present work.
Even though I write this thesis for a broad audience in mind, a basic notion in quantum physics and statistics is needed to follow it properly as I said before.
For instance, I will assume among other things that the reader knows what probability is and which its properties are, or what a quantum state is and what it represents.
I will give references where to find such complementary material when I think is necessary.

This research presented in this thesis form is part of the work done within the research group in quantum information in which Prof.
G\'eza T\'oth is the group leader and principal investigator.
I have to mention the rest of the members of the group Dr. Philipp Hyllus, Dr. Giuseppe Vitagliano, Dr. I\~nigo Urizar-Lanz, Dr. Zolt\'an Zinbor\'as and Dr. Matthias Kleinmann at the time I was working on the projects of this thesis.
Apart from the group of G\'eza T\'oth based in Bilbao, Spain, this thesis also collects some work done in collaboration with the Theoretical Quantum Optics group lead by Prof. Otfried G\"uhne at the University of Siegen, Germany, and the group of Prof. Carsten Klempt at the Leibniz University in Hannover, based also in Germany.
The last one is an experimental group specialized on the creation of exotic quantum states with very many particles with a variety of applications in quantum technology.

In this thesis, first, we investigate the metrological usefulness of a family of states known as the unpolarized Dicke states, which have boosted the sensitivity of magnetometry when lots of particles are used in the sense that quantum enhancements play a very important role to achieve such a precision.
Second, we investigate possible lower bounds on the quantum Fisher information, a quantity that characterizes the usefulness of a state for quantum metrology, using the theory of Legendre transforms such that we obtain tight lower bounds based on few measurements of the initial quantum state that will be used for metrology.
And last but not least, we investigate gradient magnetometry, i.e., we develop a theory to study the sensitivity of some states of the change in space of the magnetic field.

\begin{flushright}
  Iagoba Apellaniz

  Bilbao, \today
\end{flushright}
