In this appendix we will illustrate basic mathematical tools used all through the thesis.
They are shown here because without been figures of merit of the conceptual parts involving this thesis, they are nowadays sufficiently important for any whose intention is t expertise on this field of Quantum Metrology and Quantum Information.

\subsection{Husimi Q-representation and the Bloch sphere}

To represent states of total angular momentum bigger than $\frac{1}{2}$ we use the Husimi quasi-probability on the $\bs{n}$ unitary vector space defined as $2+2$
\be
  Q(\alpha) = \braopket{}{\varrho}{\alpha}
\ee
where ...

It is very common to express the particle density.

\subsection{Angular momentum subspaces for different spins}
\label{app:angular-subspaces}

Here I want to show how the whole Hilbert space of the spin angular-momentum of a multi-particle system splits. There are severas constituents such as the symmetric subspace, the PI subspace, the anti-symmetric subspace, etc.
