\documentclass[12pt, letterpaper, twoside]{article}

\usepackage{geometry}
\usepackage{graphicx}
\usepackage{amsmath, amsfonts, amssymb, bbm}
%\usepackage{lipsum}
\usepackage{fancyhdr}
%\usepackage{layout}
\usepackage{lettrine}
\usepackage[explicit]{titlesec}
\usepackage{hyperref}
\usepackage{watermark}
\usepackage{color}


%%%%%%%%%%%%%%%%%%%%%%%% 
% FONT SELECTION  
%%%%%%%%%%%%%%%%%%%%%%%% 

\usepackage[light]{iwona}
\usepackage[T1]{fontenc}

%%%%%%%%%%%%%%%%%%%%%%%%
% DEFINE COLORS
%%%%%%%%%%%%%%%%%%%%%%%%

\definecolor{grey}{rgb}{0.5,0.5,0.5}
\definecolor{l-grey}{rgb}{0.8,0.8,0.8}
\definecolor{withe}{rgb}{1,1,1}
\definecolor{black}{rgb}{0,0,0}

%%%%%%%%%%%%%%%%%%%%%%%% 
% PAGE LAYOUT
%%%%%%%%%%%%%%%%%%%%%%%%

% Paper Width 614pt
% Text fill symemetric 604pt
% Paper Height 794pt

\setlength{\voffset}{-0.5in}
\setlength{\hoffset}{-1in}
\setlength{\oddsidemargin}{65pt}
\setlength{\evensidemargin}{45pt}
\setlength{\topmargin}{0in}
\setlength{\headheight}{15pt}
\setlength{\headsep}{20pt}
\setlength{\marginparwidth}{0in}
\setlength{\marginparsep}{0in}
\setlength{\textheight}{635pt}
\setlength{\textwidth}{494pt}
\setlength{\footskip}{50pt}

%%%%%%%%%%%%%%%%%%%%%%%%
% TITLES LAYOUT
%%%%%%%%%%%%%%%%%%%%%%%%

\titleformat{\section}[display]
{\vspace*{150pt}
\bf\Huge}
{\begin{picture}(0,0)\put(-60,-30){\textcolor{grey}{\thesection}}\end{picture}}
{0pt}
{#1}
[]
\titlespacing*{\section}{40pt}{10pt}{40pt}[40pt]
\newcommand{\sectionbreak}{\cleardoublepage}


%%%%%%%%%%%%%%%%%%%%%%%%
% MATH OPERATORS AND 
% CUSTOM ENVIRONMENTS
%%%%%%%%%%%%%%%%%%%%%%%%
\DeclareMathOperator{\Tr}{Tr}
\DeclareMathOperator*{\Cov}{Cov}
\DeclareMathOperator{\cov}{Cov} 
  
\newcounter{observ}
\newenvironment{observ}{\refstepcounter{observ}
   \textit{\textbf{Observation \theobserv:}} \rmfamily}
     
\newenvironment{proof}{\textit{Proof:} \rmfamily}{\hfill$\square$}

\newcommand{\ket}[1]{\ensuremath{\vert #1 \rangle}}
\newcommand{\bra}[1]{\ensuremath{\langle #1 \vert}} 
\newcommand{\braket}[2]{\ensuremath{\langle #1 \vert #2 \rangle}} 
\newcommand{\braOket}[3]{\ensuremath{\langle #1 \vert #2 \vert #3 \rangle}}
\newcommand{\ketbra}[2]{\ensuremath{\vert #1 \rangle \! \langle #2 \vert}}
\newcommand{\meanO}[1]{\ensuremath{\langle #1 \rangle}}
\newcommand{\ver}[2]{\ensuremath{\genfrac{}{}{0pt}{}{#1}{#2}}}
\newcommand{\tr}[1]{\ensuremath{\Tr \lcua #1\rcua}}
\newcommand{\trsub}[2]{\ensuremath{\Tr_{#1} \lcua #2 \rcua }}
\newcommand{\bsym}[1]{\ensuremath{\boldsymbol{#1}}}

\def\be{\begin{equation}}
\def\ee{\end{equation}}
\def\bea{\begin{eqnarray}}
\def\eea{\end{eqnarray}}
\def\bse{\begin{subequations}} 
\def\ese{\end{subequations}}
\def\mtxid{\mathbbm{1}}
\def\lpar{\left(} 
\def\rpar{\right)}
\def\lcua{\left[}
\def\rcua{\right]}
\def\lcor{\left\{}
\def\rcor{\right\}}
\def\lang{\left\langle}
\def\rang{\right\rangle}
\def\l{\left} 
\def\r{\right}
\def\nnnl{\nonumber\\}
\def\nnnlq{\nonumber\\ && \quad}
\def\nnnlqq{\nonumber\\ && \qquad}
\def\nnnlqqq{\nonumber\\ && \quad\qquad}
\def\ie{, \textit{i.e.}, }

%%%%%%%%%%%%%%%%%%%%%%%%
% DOCUMENT
%%%%%%%%%%%%%%%%%%%%%%%%

\begin{document}


\pagestyle{fancy}
\renewcommand{\headrulewidth}{0pt}
\fancyhead{}
\fancyfoot{}

%%%%%%%%%%%%%%%%%%%%%%%%
% TITLE PAGE 
%%%%%%%%%%%%%%%%%%%%%%%%

%!TEX root = main.tex

\thiswatermark{\centering
\put(0,-110){\includegraphics[height=2.5cm]{img/0-Ztf.png}}
\put(430,-100){\includegraphics[height=2cm]{img/0-Ehu.png}}
}

\begin{center}

\vspace*{20pt}
\textsc{\LARGE University of the Basque Country}

\vspace{20pt}
\textsc{\Large PhD Thesis}

\vspace{50pt}
\hrule

\vspace{16pt}
{\huge \bfseries Lower bounds on quantum metrological precision}
\vspace{16pt}

\hrule
\vspace{40pt}

\begin{minipage}{0.4\textwidth}
\begin{flushleft} \large
\emph{Author:}


M. Sc. Iagoba \textsc{Apellaniz}
\end{flushleft}
\end{minipage}
\begin{minipage}{0.4\textwidth}
\begin{flushright} \large
\emph{Director:}

Prof. G\'eza \textsc{T\'oth} %TODO: Check Geza's name
\end{flushright}
\end{minipage}

\vspace{40pt}
\includegraphics[width=0.8\hsize]{img/0-cover3Dpicture.png}
\vfill

% Bottom of the page
{\large \today}

\end{center}

\cleardoublepage


%%%%%%%%%%%%%%%%%%%%%%%%
% EDITION TIPS
%%%%%%%%%%%%%%%%%%%%%%%%

\cleardoublepage

This document was generated with the 2014 distribution of \LaTeX. 

\vfill

\includegraphics[height=20pt]{img/0-CreativeCommons-by-sa.png}

2012-2015 Iagoba Apellaniz. This work is licensed under the Creative Commons
Attribution-ShareAlike 4.0 International License. To view a copy of this
license, visit
\href{http://creativecommons.org/licenses/by-sa/4.0/deed.en_US}
{http://creativecommons.org/ licenses/by-sa/4.0/deed.en\_US}.
\clearpage

%%%%%%%%%%%%%%%%%%%%%%%%
% PROLOGE
%%%%%%%%%%%%%%%%%%%%%%%%

\section*{Prologue}
\setcounter{page}{1}
\pagenumbering{roman}
\fancyfoot[LE,RO]{\thepage}

\input{prologe.tex}

%%%%%%%%%%%%%%%%%%%%%%%%
% TABLE OF CONTENTS
%%%%%%%%%%%%%%%%%%%%%%%%

\vspace*{100pt}
\tableofcontents

\section*{Tables, figures and abbreviations used in this book}
\fancyfoot[LE,RO]{\thepage}

[Insert in a table]

SLD - Symmetric logarithmic derivative.

qFI - Quantum Fisher information

%%%%%%%%%%%%%%%%%%%%%%%%
% THESIS
%%%%%%%%%%%%%%%%%%%%%%%%

\cleardoublepage

\pagenumbering{arabic}
\fancyfoot{}
%!TEX root = main.tex

\thiswatermark{\centering
\put(0,-110){\includegraphics[height=2.5cm]{img/0-Ztf.png}}
\put(430,-100){\includegraphics[height=2cm]{img/0-Ehu.png}}
}

\begin{center}

\vspace*{20pt}
\textsc{\LARGE University of the Basque Country}

\vspace{20pt}
\textsc{\Large PhD Thesis}

\vspace{50pt}
\hrule

\vspace{16pt}
{\huge \bfseries Lower bounds on quantum metrological precision}
\vspace{16pt}

\hrule
\vspace{40pt}

\begin{minipage}{0.4\textwidth}
\begin{flushleft} \large
\emph{Author:}


M. Sc. Iagoba \textsc{Apellaniz}
\end{flushleft}
\end{minipage}
\begin{minipage}{0.4\textwidth}
\begin{flushright} \large
\emph{Director:}

Prof. G\'eza \textsc{T\'oth} %TODO: Check Geza's name
\end{flushright}
\end{minipage}

\vspace{40pt}
\includegraphics[width=0.8\hsize]{img/0-cover3Dpicture.png}
\vfill

% Bottom of the page
{\large \today}

\end{center}

\cleardoublepage

\cleardoublepage
\setcounter{page}{1}

%%%%%%%%%%%%%%%%%%%%%%%%
% DEDICATION PAGE
%%%%%%%%%%%%%%%%%%%%%%%%
\vspace*{100pt}
\begin{center}
\emph{To my parents and my family}
\end{center}

\cleardoublepage

%%%%%%%%%%%%%%%%%%%%%%%%
% HEADINGS AND PAGE NUM.
%%%%%%%%%%%%%%%%%%%%%%%%

\renewcommand{\rightmark}{\bf \thesubsection}
\renewcommand{\headrulewidth}{0.5pt}
\fancyfoot[LE,RO]{\thepage}
\fancyhead[LE]{Section: \rightmark} % TODO: Solve heading
\fancyhead[RO]{\leftmark}

%%%%%%%%%%%%%%%%%%%%%%%%
% REDEFINE TITLE FORMAT
%%%%%%%%%%%%%%%%%%%%%%%%
\titleformat{\section}[display]
{\vspace*{190pt}
\bf\Huge}
{\begin{picture}(0,0)\put(-64,-31){\textcolor{grey}{\thesection}}\end{picture}}
{0pt}
{\textcolor{withe}{#1}}
[]
\titlespacing*{\section}{100pt}{10pt}{40pt}[40pt]


\section{Introduction}
\thiswatermark{\put(1,-297){\color{l-grey}\rule{84pt}{42pt}}
\put(84,-297){\color{grey}\rule{410pt}{42pt}}}

% Section: Introduction
\lettrine[lines=2, findent=3pt,nindent=0pt]{I}{n} the recent years...

The figure of merit for the precision is the inverse of the variance normalized with the number of particles, $(\Delta \Theta)^{-2}/N$. It has the following properties:

(i) The bigger it is the bigger is the precision

(ii) It is normalized so for the best separable state it is 1.
For greater values than 1 it would be a non-classical sign.

SQL

\be
  (\Delta \Theta)^{-2} \le N
\ee

HL

\be
  (\Delta \Theta)^{-2} \le N^2
\ee

This thesis consists of 4 well differentiated parts, apart from the current introduction, on which different topics are developed.
In the first part, we will introduce the reader onto the research field of quantum metrology.

Brief comments on the notation: $\text{c}_{\Theta}$ and ${\rm s}_{\Theta}$ stand for $\cos\Theta$ and $\sin\Theta$ respectively, probably some other trigonometry function is shortened.
We will omit the use of the tensor-product notation, $\otimes$, if there is not necessary for the correct comprehension of the text



\section{\foreignlanguage{nohyphenation}{Background on Estimation and Quantum Information Theories}}
\thiswatermark{\put(1,-330){\color{l-grey}\rule{84pt}{75pt}}
\put(84,-330){\color{grey}\rule{410pt}{75pt}}}

\lettrine[lines=2, findent=3pt,nindent=0pt]{T}{his} thesis is based on
many previous works developed since long time ago.
It is known that estimation processes are part of different
aspects of the human been behaviour.
From the estimation of the season on which one has 
to plant some vegetables to grow until the estimation of which route is the
shorter to reach somewhere. 
All these processes usually involve a huge amount of data. 
So it has been developed a strongly consolidated theory around that.

\input{sample_text.tex} 

\subsection{Classical estimation theory}
\input{sample_text.tex}

\subsection{Step in quantum estimation theory}
\input{sample_text.tex}

\subsection{Quantum Metrology}
\input{sample_text.tex}

\section{Quantum metrology with Dicke like states}
\thiswatermark{\put(1,-327){\color{l-grey}\rule{84pt}{72pt}}
\put(84,-327){\color{grey}\rule{410pt}{72pt}}} 
\input{sample_text.tex}

\section[Bounding qFI with observables]
{Bounding quantum Fisher \qquad\qquad\, Information with observables}
\thiswatermark{\put(1,-327){\color{l-grey}\rule{84pt}{72pt}}
\put(84,-327){\color{grey}\rule{410pt}{72pt}}} 
\input{sample_text.tex}

\section{Accuracy bound for gradient field estimation with atomic ensembles}
\thiswatermark{\put(1,-327){\color{l-grey}\rule{84pt}{72pt}}
\put(84,-327){\color{grey}\rule{410pt}{72pt}}} 
\input{sample_text.tex}

%%%%%%%%%%%%%%%%%%%%%%%%
% REDEFINE TITLE FORMAT
%%%%%%%%%%%%%%%%%%%%%%%%
\titleformat{\section}[display]
{\vspace*{150pt}
\bf\Huge}
{\begin{picture}(0,0)\put(-60,-30){\textcolor{grey}{\thesection}}\end{picture}}
{0pt}
{#1}
[]
\titlespacing*{\section}{40pt}{10pt}{40pt}[40pt]

\section*{References}
\input{sample_text.tex}

\section*{Index}
\input{sample_text.tex}

\end{document}
