This work is part of the doctoral project which I started on the sumer of 2013.
This work collects part of the research I have done on those previous to this publication fruitful years.
I will try to be as clear as possible throughout all the thesis
This way I hope it will be readable by any person with a bachelor in science, particularly in physics.
With that in mind the first and the second chapter will be used to introduce the reader into the context on which this thesis was written as well as the basic notions of Quantum Metrology, the enveloping field of the present work.
Even though I write this thesis for a broad audience in mind, a basic notion on quantum physics and statistics is needed to follow it properly.
For instance, I will assume among other things that the reader knows what probability is and which are its properties, or what a quantum state is and what it represents.
I will give references where to find such complementary material when necessary.

This research I publish in a thesis form is part of the work done within the Research Group in Quantum Information in which Prof. G\'eza T\'oth is the group leader and principal investigator.
I have to mentions the rest of the members of the group Dr. Philipp Hyllus, Dr. Giuseppe Vitagliano, Dr. I\~nigo Urizar-Lanz, Dr. Z\'oltan Z\'inboras and Dr. Matthias Kleinmann at the time I was working on the projects of this thesis. Some of them may still be part of the group and some not.
Apart from the group of G\'eza T\'oth based in Bilbao, Spain, this thesis also collects some work done in collaboration with the Theoretical Quantum Optics (TQO) group lead by Prof. Otfried G\"uhne at the University of Siegen, Germany, and the group of Prof. Carsten Klempt at the Leibniz University in Hannover, based also in Germany. The last one is an experimental group specialized on the creation of exotic quantum states for a very many particle number with a variety of applications in quantum technology.

The publication is structured such that the reader can find the thesis work after the prologue, details of my publications, table of contents and the table of abbreviations, figures and tables.
Then the thesis comes with a brief dedicatory, acknowledgements and the chapters.
The chapters are organized to follow a well structured scheme, while by coincidence a chronological order in publications is maintained too.
Finally, the conclusions and references appear at the end of this publication.

\begin{flushright}
  Iagoba Apellaniz

  Bilbao, \today
\end{flushright}
